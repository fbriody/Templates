\documentclass{extarticle}
\usepackage{amsmath}
\usepackage{tikz}
\usepackage[9pt]{extsizes}
\usepackage[margin=1.3cm,headsep=0cm]{geometry}
\usepackage[colorlinks=false]{hyperref}
\usepackage{exsheets}
\SetupExSheets{solution/print=false,question/name={},solution/name={}}	%Solutions on or off; 
\SetupExSheets{headings=runin}								% Puts questions next to number
\SetupExSheets{use-classes={easy,medium,hard}}					% Print questions by difficulty level
\SetupExSheets[points]{name=pts.}								% Puts questions next to number
\usepackage{microtype}
\usepackage{comment} %comment out blocks
\usepackage{fancyhdr}
%%%%%%%%%%%%%%%%%%%%%%%%%%%%%%
% Change these appropriately.
\newcommand{\thedate}{8/26/2012}
\newcommand{\theexam}{Quiz \S1.3 Domain and Range}
\newcommand{\thecourse}{PreCalculus $\infty$ Briody}
\newcommand{\calcnocalc}{NO CALCULATOR}
\newcommand{\theversion}{1}
\SetVariations{3}%Versions
\variant{\theversion}
%%%%%%%%%%%%%%%%%%%%%%%%%%%%%%
\fancypagestyle{nocalc}{%
	\lhead{} \chead{{\bf \Large \calcnocalc}}	\rhead{ }
	\lfoot{} \cfoot{} \rfoot{}
	\renewcommand{\headrulewidth}{0pt} \renewcommand{\footrulewidth}{0pt}
}

\begin{document}
\pagestyle{fancy}
\lhead{\thecourse} \chead{\theexam} \rhead{\thedate}
\lfoot{} \cfoot{\thepage} \rfoot{}
\renewcommand{\headrulewidth}{0.4pt} \renewcommand{\footrulewidth}{0pt}
\thispagestyle{nocalc}
\parindent 0ex
\textbf{\thecourse} \hfill  \textbf{Name:}
\makebox[6cm]{\hrulefill}

\textbf{\theexam} \hfill \textbf{\thedate}
\begin{center}
{\Large
\begin{tabular}{|l|*{\numberofquestions}{c|}c|}\hline
%  Question & \ForEachQuestion{\QuestionNumber{#1}\iflastquestion{}{&}} & Total \\ \hline
  Question & \ForEachQuestion{\GetQuestionProperty{counter}{#1}\iflastquestion{}{&}} & Total \\ \hline
  Points   & \ForEachQuestion{\GetQuestionProperty{points}{#1}\iflastquestion{}{&}} & \pointssum* \\ \hline
  Score  & \ForEachQuestion{\iflastquestion{}{&}} & \\ \hline
\end{tabular}
}
\end{center}
\rule[1ex]{\textwidth}{.1pt}
{\large Please produce {\bf neat} and {\bf organized} solutions. Be sure to \framebox{box} your answers when appropriate. Messy or unsupported answers will receive no credit. If you would like scratch paper please ask. Good luck.}\\
\rule[1ex]{\textwidth}{.1pt}
%%%%%%%%%%%%%%%%%%QUESTIONS%%%%%%%%%%%%%%%%%%%%%%
\begin{question}[class=medium]\addpoints*{2}
Find the domain of $f(x)=\frac{1}{x-\vary{2}{4}{6}}$. 
\end{question}
\begin{solution}
$(-\infty,\vary{2}{4}{6})(\vary{2}{4}{6},\infty)$
\end{solution}

\begin{question}[class=medium]\addpoints*{4}
Find the domain and range of $g(x)=\sqrt{\vary{3}{5}{7}-x}+\vary{2}{4}{6}$.
\end{question}
\begin{solution}
Domain: $(-\infty,\vary{3}{5}{7})$ Range:$[\vary{2}{4}{6},\infty]$
\end{solution}


\begin{question}[class=medium]\addpoints*{3}
Find the domain of $h(x)=\frac{1}{x-\vary{4}{6}{8}}+\sqrt{x}$.
\end{question}
\begin{solution}
$[0,\vary{4}{6}{8})\cup(\vary{4}{6}{8},\infty)$ 
\end{solution}


%%%%%%%%%%%%%%%%%%QUESTIONS%%%%%%%%%%%%%%%%%%%%%%
\newpage
Answers for version \theversion
\printsolutions
\end{document}



