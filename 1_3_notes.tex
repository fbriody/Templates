\documentclass[9pt, oneside]{extarticle}   	% use "amsart" instead of "article" for AMSLaTeX format
%\documentclass[9pt,twocolumn,oneside]{extarticle}   	% use for two column
\usepackage{graphicx}				% Use pdf, png, jpg, or eps� with pdflatex; use eps in DVI mode
								% TeX will automatically convert eps --> pdf in pdflatex								
\usepackage[margin=2cm]{geometry}
\usepackage{amssymb}
\usepackage{tikz,adjust box}			%adjustbox is for figures that span 2 columns
\usepackage{dashrule}				%used to make dashed line in text
\usepackage{amsmath}
\usepackage{multicol}
\usepackage{fancyhdr}				%used to put draft in header
\DeclareMathOperator{\arcsec}{arcsec}	%arcsec, arccsc... not defined inamsmath
\usepackage{exsheets}				%Formats questions and solutions
\SetupExSheets{solution/print=false,question/name={},solution/name={}}		%Solutions on or off; 
\SetupExSheets{headings=runin}		% Puts questions next to number
\SetupExSheets{use-classes={easy,medium,hard}}	% Print questions by difficulty level
%\usepackage{natbib,endnotes}			%Bibliography and Footnote to endnote
\usepackage{title sec}				%Format section titles to include horizontal line
\usepackage{enumitem}				%Allow lists to be a, b, c and not 1, 2, 3

\setenumerate[0]{label=(\alph*)}
\let\footnote=\endnote
\titleformat{\section}
  {\normalfont\Large\bfseries}{\thesection}{1em}{}[{\titlerule[0.8pt]}]
\newcommand{\calc}{\protect\includegraphics[height = 2.5ex]{images/calc.png}}  

\newcommand{\thedate}{8/25/2014}
\newcommand{\theexam}{Notes \S 1.3 Domain and Range}
\newcommand{\thecourse}{PreCalculus $\infty$ Briody}

%%%%%%Macros: begin question cntrl command Q or S for solution
\title{Problem Solving}
\author{PHS}
\date{today}							% Activate to display a given date or no date
%\chead{draft\_draft\_draft\_draft\_draft\_draft}
\begin{document}
\pagestyle{fancy}
\lhead{\thecourse} \chead{\theexam} \rhead{\thedate}
\lfoot{} \cfoot{\thepage} \rfoot{}
\renewcommand{\headrulewidth}{0.4pt} \renewcommand{\footrulewidth}{0.4pt}
\thispagestyle{plain}

\parindent 0ex
\textbf{\thecourse} \hfill  \textbf{Name:}
\makebox[6cm]{\hrulefill}

\textbf{\theexam} \hfill \textbf{\thedate}
\rule[1ex]{\textwidth}{.1pt}
{\large {\bf Objectives} You should be able to find the allowed inputs ({\bf domain}) and resulting outputs ({\bf range}) for given functions. Be especially careful with functions that contain fractions and/or square roots. Please use interval notation.\\
\hrule 
\kern3pt
%\section{Functions}
%\input{subsections/func} 
\begin{question}[class=easy]
Suppose $p(x)=2\sqrt{x}+3$.
\begin{enumerate}
\item Is there a value of $x$ for which $p(x)=5$?
\item What is the domain of $p$?
\item \label{range} \calc\ What is the range of $p$? Explain. 
\item \calc\ Does the answer to \ref{range} change if  $4\leq x\leq9$? Explain.
\end{enumerate}
\end{question}
\begin{solution}
\begin{enumerate}
\item $x=1$
\item $[0,\infty)$
\item $[3,\infty)$
\item $[7,9]$
\end{enumerate}
\cite[p.6]{AOPS}
\end{solution}

\begin{question}[class=medium]
Find the domain (and range?) of each of the following functions:
\begin{enumerate}
\item $A(x)=4x^{2}+1$
\item $o(x)=3+\sqrt{16-(x-3)^{2}}$
\item $P(x)=\frac{1}{3+\sqrt{x+1}}$
\item $S(x)=\frac{12x-9}{6-9x}$
\end{enumerate}
\end{question}
\begin{solution}
\begin{enumerate}
\item Domain: $\mathbb{R}$ Range: $[1,+\infty)$
\item Domain: $[-1,7]$ Range: $[3,7]$
\item Domain: $[-1,+\infty)$ Range: $(0,1/3]$
\item Domain: All $\mathbb{R}$ except 2/3 Range: All $\mathbb{R}$ except -4/3
\end{enumerate}� \cite[p.26]{AOPS}
\end{solution}

\begin{question}[class=medium]
Find the domain of $g(t)=\frac{2t-4}{\frac{1}{t}-\frac{1}{3t-4}}$.
\end{question}
\begin{solution}
$[5/2,3)\cup (3,+\infty)$ \cite[p.7]{AOPS}
\end{solution}

\begin{question}[class=medium]
Find the domain of each of the following functions:
\begin{enumerate}
\item $f(x)=\frac{1}{\sqrt{2x-5}}+\sqrt{9-3x}$
\item $f(x)=|\sqrt{x}-2|+|\sqrt{x-2}|$
\item $g(x)=\sqrt{|x|-2}+\sqrt{|x-3|}$
\end{enumerate}�
\end{question}
\begin{solution}
\begin{enumerate}
\item $(5/2,3]$
\item $[2,+\infty)$
\item $(-\infty,-2]\cup [2,\infty)$
\end{enumerate}� \cite[p.26]{AOPS}
\end{solution}
\newpage
\begin{multicols}{2}
\section{SOLUTIONS}
\printsolutions
\bibliographystyle{te}
\bibliography{temp}
\end{multicols}
%\theendnotes
\end{document}